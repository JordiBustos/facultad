
\documentclass{article}

\usepackage[margin=1in]{geometry} 
\usepackage{amsmath,amsthm,amssymb}
\usepackage[spanish]{babel}
\usepackage{newpxtext,newpxmath}

\newcommand{\R}{\mathbb{R}}
\newcommand{\Z}{\mathbb{Z}}
\newcommand{\N}{\mathbb{N}}
\newcommand{\Q}{\mathbb{Q}}
\newcommand{\C}{\mathbb{C}}

\newenvironment{theorem}[2][Ejercicio]{\begin{trivlist}
\item[\hskip \labelsep {\bfseries #1}\hskip \labelsep {\bfseries #2.}]}{\end{trivlist}}
\newenvironment{lemma}[2][Lemma]{\begin{trivlist}
\item[\hskip \labelsep {\bfseries #1}\hskip \labelsep {\bfseries #2.}]}{\end{trivlist}}
\newenvironment{exercise}[2][Exercise]{\begin{trivlist}
\item[\hskip \labelsep {\bfseries #1}\hskip \labelsep {\bfseries #2.}]}{\end{trivlist}}
\newenvironment{problem}[2][Problem]{\begin{trivlist}
\item[\hskip \labelsep {\bfseries #1}\hskip \labelsep {\bfseries #2.}]}{\end{trivlist}}
\newenvironment{question}[2][Question]{\begin{trivlist}
\item[\hskip \labelsep {\bfseries #1}\hskip \labelsep {\bfseries #2.}]}{\end{trivlist}}
\newenvironment{corollary}[2][Corollary]{\begin{trivlist}
\item[\hskip \labelsep {\bfseries #1}\hskip \labelsep {\bfseries #2.}]}{\end{trivlist}}

\newenvironment{solution}{\begin{proof}[Solution]}{\end{proof}}

\begin{document}

\title{Geometría Diferencial}
\author{Bustos Jordi\\Práctica I}

\maketitle

\begin{theorem}{1}
  Dada una curva \( \alpha : I \to \R^3 \), parametrizada por longitud de arco, con curvatura y torsión no nulas tal que
  \( \| \alpha(s) \| = a \), para todo \( s \in I \). Demostrar que se cumple la siguiente ecuación: \begin{align*}
    {\left( \frac{1}{\kappa} \right)}^2 + \left( \frac{\dot{\kappa}}{\kappa^2 \tau} \right) = a^2
  \end{align*}
\end{theorem}

\begin{proof}
  Como \( \| \alpha(s) \| = a \) constante, diferenciando \( \alpha \cdot \alpha = a^2 \) obtenemos \begin{align*}
    2 \alpha \cdot \dot{\alpha} = 0 \implies \alpha \cdot t(s) = 0 \quad \forall s \in I
  \end{align*}
  es decir, \( t(s) \) es ortogonal a \( \alpha(s) \) para todo \( s \in I \). Usando la base de Frenet \( \{ t, n, b\} \) podemos escribir \begin{align*}
    \alpha(s) = A(s) n(s) + B(s) b(s)
  \end{align*}
  Diferenciando y utilizando las fórmulas de Frenet \begin{align*}
    t' & = \kappa n           \\
    n' & = -\kappa t + \tau b \\
    b' & = -\tau n
  \end{align*}
  y recordando que \( t = \alpha' \) por estar parametrizada por longitud de arco, obtenemos \begin{align*}
    t = \dot{\alpha} & = A' n + A n' + B' b + B b'                      \\
                     & = (-A\kappa) t + (A' - B\tau) n + (B' + A\tau) b
  \end{align*}
  Comparando componentes \begin{align*}
    1 + A\kappa & = 0 \\
    A' - B\tau  & = 0 \\
    B' + A\tau  & = 0
  \end{align*}
  De la primera ecuación obtenemos \( A = \frac{-1}{\kappa} \) pues la curvatura es no nula y derivando \( A' = \frac{\dot{\kappa}}{\kappa^2} \). Sustituyendo en la segunda ecuación \begin{align*}
    B & = \frac{\dot{\kappa}}{\kappa^2 \tau} \quad \tau \neq 0 \text{ por hipótesis}
  \end{align*}
  Luego, \begin{align*}
    \| \alpha(s) \|^2 & = {A(s)}^2 + {B(s)}^2                                                                         \\
                      & = {\left( \frac{1}{\kappa} \right)}^2 + {\left( \frac{\dot{\kappa}}{\kappa^2 \tau} \right)}^2 \\
                      & = a^2
  \end{align*}
  que es lo que queríamos demostrar.
\end{proof}

% -----------------------------------------------
% Ignore everything that appears below this.
% -----------------------------------------------

\end{document}